\chapter{توصیف معماری سیستم}
\noindent
\textbf{
\textit{
تشریح اینترفیس‌های سیستم، کلاک‌ها و نحوهٔ راه‌اندازی سیستم، دیاگرام بلوکی سخت‌افزار، ساختار درختی سیستم و توصیف ماژول‌های سخت‌افزار 
}
}
\pagebreak

\section{ اینترفیس‌های سیستم}
در ابتدا به صورت خلاصه اینترفیس‌های سیستم سخت‌افزاری الگوریتم Skein بیان می‌شود، اینترفیس یک سیستم شامل ورودی‌ها و خروجی‌ها و مشخصات ایشان است. 
\subsection{ورودی‌ها}
ورودی‌ها کد verilog الگوریتم Skein به شرح زیر اند.
\begin{itemize}
\item
\textbf{clk}\\
ورودی کلاک سیستم است که با آن سیستم کار خود را به صورت ترتیبی 
\footnote{\lr{Sequential}}
انجام می‌دهد، فرکانس کلاک با توجه به نحوهٔ پیاده‌سازی سخت‌افزاری و نتایج حاصل از سنتز تعیین می‌شود.
\item
\textbf{\lr{midstate}}\\
ورودی‌ای ۵۱۲ بیتی برای الگوریتم
\lr{Skein-512}
  است که حالت میانی در هش را معلوم می‌کند.
\item
\textbf{nonce}\\
nonce مقداری دلخواه است که برای به حداکثر رساندن تصادفی  و غیرقابل شکستن بودن هش 
در محاسبه هش استفاده می‌شود، این مقدار می‌تواند عددی دلخواه باشد. در الگوریتم 
\lr{Skein-512}
 اندازهٔ این ورودی ۳۲ بیت به اندازه طول عدد در Integer گرفته شده است.
\item
\textbf{data}\\
ورودی اصلی‌ست که باید هش آن محاسبه شود، در کد verilog داده شده اندازه این ورودی ۹۶ بیت در نظر گرفته شده است. 
\end{itemize}

\subsection{خروجی}
تنها خروجی سیستم مقدار هش در output است که ۵۱۲ بیت طول دارد.
(الگوریتم مورد بحث 
\lr{Skein-512}
 است)

\subsection{کلاک‌ها و نحوهٔ راه‌اندازی سیستم}
این سیستم فقط از یک کلاک استفاده می‌کند و برای راه‌اندازی سیستم انجام کارهای زیر ضروری‌ست.
\begin{enumerate}
\item
وصل کردن کلاک با فرکانس مناسب به سیستم
\item 
تعیین ورودی‌های اولیه 
\item 
راه‌اندازی سیستم 
\end{enumerate}

