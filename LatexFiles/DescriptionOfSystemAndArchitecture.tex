\chapter{توصیف معماری سیستم}
\noindent
\textbf{
	\textit{
		تشریح اینترفیس‌های سیستم، کلاک‌ها و نحوهٔ راه‌اندازی سیستم، دیاگرام بلوکی سخت‌افزار، ساختار درختی سیستم و توصیف ماژول‌های سخت‌افزار 
	}
}
\pagebreak

\section{ اینترفیس‌های سیستم}
در ابتدا به صورت خلاصه اینترفیس‌های سیستم سخت‌افزاری الگوریتم Skein بیان می‌شود، اینترفیس یک سیستم شامل ورودی‌ها و خروجی‌ها و مشخصات ایشان است. 
\subsection{ورودی‌ها}
ورودی‌ها کد verilog الگوریتم Skein به شرح زیر اند.
\begin{itemize}
	\item
	      \textbf{clk}\\
	      ورودی کلاک سیستم است که با آن سیستم کار خود را به صورت ترتیبی 
	      \footnote{\lr{Sequential}}
	      انجام می‌دهد، فرکانس کلاک با توجه به نحوهٔ پیاده‌سازی سخت‌افزاری و نتایج حاصل از سنتز تعیین می‌شود. در Testbench داده شده کلاک هر ۱۰ نانوثانیه تغییر می‌کند.
	\item
	      \textbf{\lr{midstate}}\\
	      ورودی ۵۱۲ بیتی برای الگوریتم
	      \lr{Skein-512}
	      است که حالت میانی در هش را معلوم می‌کند.
	\item
	      \textbf{nonce}\\
	      nonce مقداری دلخواه است که برای به حداکثر رساندن تصادفی  و غیرقابل شکستن بودن هش 
	      در محاسبه هش استفاده می‌شود، این مقدار می‌تواند عددی دلخواه باشد. در الگوریتم 
	      \lr{Skein-512}
	      اندازهٔ این ورودی ۳۲ بیت به اندازه طول عدد در Integer گرفته شده است.
	\item
	      \textbf{data}\\
	      ورودی اصلی‌ست که باید هش آن محاسبه شود، در کد verilog داده شده اندازه این ورودی ۹۶ بیت در نظر گرفته شده است. 
\end{itemize}

\subsection{خروجی}
تنها خروجی سیستم مقدار هش در output است که ۵۱۲ بیت طول دارد.
(الگوریتم مورد بحث 
\lr{Skein-512}
است)

\section{کلاک‌ها و نحوهٔ راه‌اندازی سیستم}
این سیستم فقط از یک کلاک استفاده می‌کند و برای راه‌اندازی سیستم انجام کارهای زیر ضروری‌ست.
\begin{enumerate}
	\item
	      وصل کردن کلاک با فرکانس مناسب به سیستم
	\item
	      اعمال ریست‌ کلی بر سیستم
	      \footnote{\lr{Global Reset}}
	\item 
	      تعیین ورودی‌های اولیه 
	\item 
	      راه‌اندازی سیستم 
\end{enumerate}


\section{دیاگرام بلوکی سخت‌افزار}
دیاگرام بلوکی کلی سخت‌افزار در شکل 
\ref{block_diagram}
آمده است. 

\begin{figure}
	\includegraphics[width = \textwidth]{figs/DescriptionOfSystem/block_diagram.jpg}
	\caption{دیاگرام بلوکی سخت‌افزار}
	\label{block_diagram}
\end{figure}

\section{توصیف ماژول‌های سخت‌افزار}
\subsection{\lr{Skein-512}}
\subsubsection{اطلاعات کلی}
\lr{
\begin{table}[H]
\centering
\begin{tabular}{@{}ll@{}}
\toprule
\multicolumn{2}{l}{skein512}                                           \\ \midrule
$clk, {[}511:0{]} midstate, {[}95:0{]} data, {[}31:0{]} nonce$ & inputs  \\
${[}511:0{]} hash $                                            & outputs \\ \bottomrule
\end{tabular}%
\rl{
\caption{اطلاعات کلی ماژول \lr{skein512}}}
\label{table_skein512}
\end{table}}
درخطوط اولیه تعداد reg و wire تعریف شده است.
دو reg به نام های$ phase\_d$ و $phase\_q $تعریف شده اند که یک‌بیتی اند و مقدار صفر به آنها داده شده است.\\
دو assignment یک‌خطی دیده می‌شود.

\begin{enumerate}
	\item در reg 32 بیتی با نام $nonce\_le$ که در خطوط بالاتر تعریف شده است مقادیر $nonce$ (که ورودی 32 بیتی ماژول هستند) به صورت 8 بیت – 8 بیت و به صورت برعکس ذخیره می‌شوند. یعنی به طور مثال 8 بیت کم ارزش\  $nonce$در 8 بیت پرارزش $nonce\_le$ ذخیره شده اند.
	      
	\item در reg 32 بیتی با نام $nonce2\_le$ که در خطوط بالاتر تعریف شده است مقادیر $nonce2$ (که برعکس $nonce$، ورودی ماژول نیست و خود در خطوط بالاتر به صورت یک reg 32 بیتی تعریف شده است و در واقع در حال حاضر مقداری را به خود اختصاص نداده است) به صورت 8 بیت – 8 بیت و به صورت برعکس ذخیره می‌شوند. یعنی به طور مثال 8 بیت کم ارزش\  $nonce2$در 8 بیت پرارزش  $nonce2\_le$ ذخیره شده اند. (خط 56)
\end{enumerate}

یک عبارت assign طویل مربوط به hash دیده می‌شود:

\begin{itemize}
	\item
	      در این عبارت بیت‌های reg ی به نام
	      h\_q 
	      که 512 بیت دارد و در خطوط بالاتر تعریف شده است، به بیت های خروجی hash اساین می‌شود.
	\item
	      64 مجموعه 8 بیتی از $h\_q$ به بیت های hash اساین می‌شود که نظم این مقداردهی در زیر توضیح داده می‌شود. در این توضیحات hash را به ترتیب از پرارزش‌ترین ۸ بیت شروع به پر کردن میکنیم.
	\item
	      پرارزش‌ترین بیت‌های hash با بیت های $463$ تا $456$ پر شده است. (یعنی پرارزش ترین بیت hash با بیت $463$ ام $h\_q$ پر شده است و به همین ترتیب)
	\item
	      مجموعه بعدی 8 تایی از$464$ تا $471$ هستند که در دومین 8 تایی با ارزش hash قرار می‌گیرند.
	\item
	      این روند تا هشتمین 8 بیت ارزشمند hash ادامه پیدا میکند جایی که در این جایگاه مجموعه $[511:504]$ از $h\_q$ جای میگیرد. (تا اینجا نظم داشتیم)
	\item
	      نهمین 8 بیت ارزشمند hash توسط بیت های $[391:384]$ از $h\_q$ پر میشوند.
	\item
	      این روند ادامه پیدا میکند (یعنی دهمین 8 بیت ارزشمند با $[399:392]$ پر میشوند).
	      تا 16امین 8 بیت ارزشمند hash که با مجموعه $[447:440]$ پر شده اند.
	\item
	      17امین 8 بیت ارزشمند با مجموعه $[327:320]$ پر میشود.
	\item
	      این روند مانند قبل به صورت صعودی ادامه پیدا خواهد کرد تا به 25 امین مجموعه 8 بیتی برسیم.
	      
\end{itemize}

\textit{\textbf{درواقع هر 8 بار که مجموعه بیت های 8 بیتی را assign میکنیم، یک بی‌نظمی داریم.
}}
\begin{itemize}
	\item
	      25 امین 8 بیتی hash با بیت های $[263:256]$ پر میشود.
	\item
	      دوباره روند سابق و صعودی را داریم تا به 33 امین assignment برسیم.
	\item
	      33 امین 8 بیتی hash با بیت های $[199:192]$
	      
\end{itemize}

\textit{\textbf{هر بار بی‌نظمی داریم بازه جدید بعد از بی نظمی 120 واحد کمتر از بازه قبلی خواهد بود مثلا 32 امین 8بیت پرارزش hash با بیت های $[319:312] $پر شده اند که 120 واحد از بازه ای که برای 33امین 8 بیت ارزشمند hash اختصاص داده میشود بیشتر است. (در بالا 33امین نوشته شده است)}}

\begin{itemize}
	\item
	       8 مجموعه که به صورت صعودی پیش برویم به 40 امین 8بیت میرسیم که طبق نظم با بیت های $[255:248]$ پر شده است و 41امین 8 بیتی با بازه $[135:128]$ پر شده است.
	\item
	       8 مجموعه که به صورت صعودی پیش برویم به 48 امین 8بیت میرسیم که طبق نظم با بیت های $[191:184]$ پر شده است و 49امین 8 بیتی با بازه $[71:64]$ پر شده است.
	\item
	       8 مجموعه که به صورت صعودی پیش برویم به 56 امین 8بیت میرسیم که طبق نظم با بیت های $[127:120]$ پر شده است و 57امین 8 بیتی با بازه $[7:0]$ پر شده است.
	\item
	      از 57 امین مجموعه 8 تایی با ارزش hash تا آخرین مجموعه باارزش hash (64 امین) نیز به صورت صعودی و طبق نظم پیش میرود. (خط 121)
	      
\end{itemize}

بعد از خطوط 
assignment،
 ۱۸ instance از ماژول skein\_round گرفته شده است.
این instance ها را از 
$00$
تا
$0H$
نام گذاری کردیم (نامگذاری در مبنای بالاتر از 10 شده است)

\subsubsection{
	ورودی skein\_round ها
}
\begin{itemize}
	\item
	      \textbf{کلاک}
	      که همه به کلاک سیستم متصل اند.
	\item
	      \textbf{Round}
	      رجیستر 32 بیتی که به ترتیب ورودی 0 تا 17 به هر اینستنس داده شده است.
	\item
	      \textbf{$\textbf{p}$} 
	      رجیستر 512 بیتی – که به اینستنس شماره $01$ تا $0H$ به ترتیب $p01$ تا $p0H$ وصل شده است. به اینستنس شماره $00$ هم $p00\_q$ وصل شده است.
	\item
	      \textbf{$\textbf{H}$}
	      رجیستر 576 بیتی – که به اینستنس شماره $01$ تا $0H $به ترتیب $h01$ تا $h0H$ وصل شده است. به اینستنس شماره $00$ هم $h00\_q$ وصل شده است.
	\item
	      \textbf{$\textbf{T0}$}
	      رجیستر 64 بیتی –   که به اینستنس شماره $00$ تا $0H$ به ترتیب این دنباله 3 تایی وصل شده است:
	      $t0\_q, t1\_q, t2\_q$
	      این دنباله 3 جمله ای به ترتیب تکرار میشود.
	\item
	      \textbf{$\textbf{T1}$}
	      رجیستر 64 بیتی – دقیقا مثل $T0$ با این تفاوت که دنباله 3 تایی $ t1\_q، t2\_q, t0\_q $ به این شکل است.
	\item
	      \textbf{$\textbf{P0}$}
	      رجیستر 512 بیتی- که اینستنس شماره $00$ تا $0H$به ترتیب $o00$ تا $o0H$ وصل شده است. 
	\item
	      \textbf{$\textbf{H0}$}
	      رجیستر 576 بیتی- که اینستنس شماره $00$ تا $0H $به ترتیب $ho00 $تا $ho0H$ وصل شده است. خط(141)
	      
\end{itemize}
\textit{\textbf{
	در ادامه یک always بلاک داریم که حساس به تغییرات همه چیز است. (خط 143)
	در این بلاک متغیر هایی که در انتهایشان
	\_d
دارند مقداردهی میشوند.}}\\
ابتدا phase\_d مقدار not متغیر phase\_q را به خود اختصاص میدهد.
\begin{itemize}
	\item
	      \textbf{اگر phase\_q یک باشد}
	      \begin{itemize}
	      	\item
	      	      مقداردهی به $p00\_d$ (512 بیتی): 64 بیت کم ارزش ( $[63:0]$) از data در 64 بیت پرارزش $p00\_d$ قرار میگیرد. سپس در 32 بیت بعدی $p00\_d$ (از چپ) عینا nonce\_le قرار داده میشود. سپس 32 بیت باقیمانده از $data ([95:64])$ طبق روند قرار داده میشود. باقی بیت های این رجیستر هم با صفر پر میشوند $(384'd)$ – (خط 148)
	      	\item
	      	      مقداردهی به$ h00\_d$ (576 بیتی): در 64 بیت کم ارزش این reg مقدار صفر قرار داده میشود و باقی بیت ها دقیقا به midstate (ورودی 512 بیتی) متصل میشوند.
	      	\item
	      	      مقداردهی به$ t0\_d$ (64 بیتی) : $h0000000000000050$
	      	\item
	      	      مقداردهی به$ t1\_d$ (64 بیتی) : $hb000000000000000$
	      	\item
	      	      مقداردهی به $t2\_d$ (64 بیتی) : $hb000000000000050$
	      	\item
	      	      h\_d هم مقدار h\_q را به خود میگیرد.
	      \end{itemize}
	\item
\textbf{	      اگر phase\_q صفر باشد
}	      \begin{itemize}
	      	\item
	      	      مقداردهی به $p00\_d$ (512 بیتی): این reg با صفر پر میشود.
	      	\item
	      	      مقداردهی به $h00\_d$ (576 بیتی):
	      	      \begin{itemize}
	      	      	\item
	      	      	      بیت های $[575:512]$:	
	      	      	      	$data[63:0] \textasciicircum ( oH[511:448] + hH[575:512])$
	      	      	\item
	      	      	      بیت های $[511:448]$: 
	      	      	      	\{ $nonce2\_le$ , $data[95:64$ \} \textasciicircum ( $oH[447:384] + hH[511:448]$)
	      	      	\item
	      	      	      بیت های $[447:384]$:  $	oH[383:320] + hH[447:384]$
	      	      	\item
	      	      	      بیت های $[383:320]$:$	oH[319:256] + hH[383:320]$
	      	      	\item
	      	      	      بیت های $[319:256]$:$	oH[255:192] + hH[319:256]$
	      	      	\item
	      	      	      بیت های $[255:192]$:
	      	      	      $	oH[191:128] + hH[255:192] + 
	      	      	      64'h0000000000000050$
	      	      	\item
	      	      	      بیت های $[191:128]$:	
	      	      	      $oH[127: 64] + hH[191:128] + 64'hb000000000000000$
	      	      	\item
	      	      	      بیت های $[127:64]$:
	      	      	  $    	oH[ 63:  0] + hH[127: 64] + 18$
	      	      \end{itemize}
	      	\item
	      	      مقداردهی به$ t0\_d$ (64 بیتی) : $h0000000000000008$
	      	\item
	      	      مقداردهی به $t1\_d$ (64 بیتی) : $hFF00000000000000$
	      	\item
	      	      مقداردهی به $t2\_d$ (64 بیتی) : $hFF00000000000008$
	      	\item
	      	      مقداردهی به $h\_d$ (512 بیتی):
	      	      \begin{itemize}
	      	      	\item
	      	      	      بیت های $[511:448]$:
	      	      	      $ 	o0H[511:448] + ho0H[575:512]$
      	      	    \item
	      	      	      بیت های $[447:384]$:  
	      	      	      	$o0H[447:384] + ho0H[511:448]$
	      	      	\item
	      	      	      بیت های $[383:320]$:
	      	      	      $	o0H[383:320] + ho0H[447:384]$
	      	      	\item
	      	      	      بیت های $[319:256]$:
	      	      	      $	o0H[319:256] + ho0H[383:320]$
	      	      	\item
	      	      	      بیت های $[255:192]$:
	      	      	      $	o0H[255:192] + ho0H[319:256]$
	      	      	\item
	      	      	      بیت های $[191:128]$:	
	      	      	      $o0H[191:128] + ho0H[255:192] + 64'h0000000000000008$
	      	      	\item
	      	      	      بیت های $[127:64]$:	
	      	      	      $o0H[127: 64] + ho0H[191:128] + 64'hFF00000000000000$
	      	      	\item
	      	      	      بیت های $[63:0]$:	
	      	      	      $	o0H[ 63:  0] + ho0H[127: 64] + 18$
	      	      	 
	      	      \end{itemize}
	      \end{itemize}
\end{itemize}
\textit{
	\textbf{نظم مناسبی دیده میشود به این شکل که به ترتیب 64 بیت پرارزش h\_d با مجموع 64 بیت پرارزش $o0H$ و 64 بیت پرارزش $ho0H$ پر میشود. به جز 3 مورد آخر که با اعدادی ثابت هم جمع میشوند.
}}\\

\textit{\textbf{Always بلاک دوم فقط به لبه مثبت کلا384’ک حساس است. (خط 211)
	(عموما متغیر های \_q، مقادیر متناظر \_d را به خود میگیرند) }}
\begin{itemize}
	\item
	      $hH$ مقدار $ho0H$ را به خود میگیرد.
	\item
	      $oH$ مقدار $o0H$ را به خود میگیرد.
	\item
	      $phase\_q$ مقدار $phase\_d$ را به خود میگیرد.
	\item
	      $h\_q$ مقدار $h\_d$ را به خود میگیرد.
	\item
	      $t0\_q$ مقدار $t0\_d$ را به خود میگیرد.
	\item
	      $t1\_q$ مقدار $t1\_d$ را به خود میگیرد.
	\item
	      $t2\_q$ مقدار $t2\_d$ را به خود میگیرد.
	\item
	      در ادامه مجموعه ای از مقدار دهی ها را مربوط به reg های
	       $p0x$
	      و 
	      $h0x$ 
	داریم. (خط 226 تا 261)
	( H منظور از ۱ تا H )
	\item
	      $h0x$ ها: مقدار $ho0y$ را میگیرند با این تفاوت که y از x یک واحد کمتر است. (به طور مثال $h01$ مقدار $ho00$ را به خود میگیرد)
	\item
	      $p0x$  ها: مقدار $o0y$ را میگیرند با این تفاوت که y از x یک واحد کمتر است. (به طور مثال $h01$ مقدار $o00$ را به خود میگیرد)
	\item
	      $p00\_q$ مقدار $p00\_d$ را میگیرد.
	\item
	      $h00\_q$ مقدار $h00\_d$ را میگیرد.
	\item
	      $nonce2$ هم که در ابتدای فایل مقداری مجهول داشت اینجا مقدار $nonce$ (ورودی)  منهای 
	      $32’d54$
	       را میگیرد.
\end{itemize}


\subsection{skein\_round}
در خط 124 از فایل وریلاگ، 18 اینستنس از ماژول skein\_round گرفته شده است.
\subsubsection{اطلاعات کلی}
\lr{
\begin{table}[H]
\centering
\begin{tabular}{@{}ll@{}}
\toprule
\multicolumn{2}{l}{skein\_round}                                                            \\ \midrule
$clk, {[}31:0{]} round, {[}511:0{]} p, {[}575:0{]} h, {[}63:0{]} t0, {[}63:0{]} t1 $& inputs  \\
${[}511:0{]} p0, {[}575:0{]} h0 $                                                   & outputs \\ \bottomrule
\end{tabular}%
\rl{
\caption{اطلاعات کلی ماژول \lr{skein\_round}}}
\label{table_skein_round}
\end{table}}
\begin{itemize}
\item
در این ماژول چهار ماژول دیگر زیر ایجاد شده اند. 
\begin{itemize}
\item
\lr{skein\_round\_1}
\item
\lr{skein\_round\_2}
\item
\lr{skein\_round\_3}
\item
\lr{skein\_round\_4}
\end{itemize}
\item
در این ماژول یک 
\lr{always block}
و تعدادی 
\lr{assignment}
وجود دارد.
\item
دو مجموعه reg تعریف شده:
\begin{itemize}
\item
64 بیتی: $p0, p1, p2, p3, p4, p5, p6, p7$
\item
576 بیتی: $hx0, hx1, hx2, hx3, hx4$
\end{itemize}
\item
یک مجموعه wire تعریف شده:
\begin{itemize}
\item
512 بیتی: $po0, po1, po2, po3, po4$
\end{itemize}
\item
3 عدد assignment داریم:
\begin{itemize}
\item
 $ho$ (یکی از خروجی ها) از $hx4$ مقدار میگیرد. 
 \item
$po$ (دیگر خروجی) از $po4$ مقدار میگیرد. 
\item
$Po0$ به ترتیب 8 بیت-8بیت (ازبا  ارزش به کم ارزش) از reg های $p0, p1, ..., p7$ مقدار میگیرد.
\end{itemize}
\item
از هر 4 ماژول باقی مانده
\lr{(skein\_round\_1,2,3,4)}
 در کد یک اینستنس گرفته شده است. (خط 310)
 
 \begin{itemize}
 \item ورودی کلاک به کلاک سیستم متصل شده است.
ورودی$ even$ ماژول ها همگی به  $!round[0]$ متصل اند. (یکی از بیت های ورودی)
\item
به عنوان $in$ و$ out $ هم به هر ماژول $po(x)$ و $po(x+1)$ داده میشود که $x+1$ شماره$ round$ ماژول است. به طور مثال به ماژول $skein\_round\_1$ برای ورودی $po0$ و برای خروجی $po1$ داده میشود.
 \end{itemize}
\textit{\textbf{ نکته مهم این است که خروجی ماژول 1 ورودی ماژول 2 است و به همین ترتیب تا ماژول ۴.
}}
\item
یک always-block حساس به لبه مثبت کلاک داریم. (خط 315)
\begin{itemize}
\item
ورودی های $h $و $p $را به صورت 64 بیت – 64 بیت جمع میزند و در $p0$ تا $p7$نگه داری میکند.
\item
به این شکل که جمع پرارزش ترین64 بیت $h$ و $p$ در $p0$ ریخته میشود. ( و به همین ترتیب پیش میرود)
\item
از $p0$ تا $p4$ کاملا طبق نظم گفته شده انجام میشود.
\item
در مورد $p5$ علاوه بر دو مجموعه 64 بیتی با $t0$ (یکی از ورودی ها) هم جمع میشود.
\item
در مورد $ p6$ علاوه بر دو مجموعه 64 بیتی با$ t1$ (یکی از ورودی ها) هم جمع میشود.
\item
در مورد $ p7 $علاوه بر دو مجموعه 64 بیتی با $round$ (یکی از ورودی ها) هم جمع میشود.
\end{itemize}
\item
برای reg های hx (576 بیتی) هم یک جابه جایی اتفاق میافتد:
\begin{itemize}
\item
به این شکل که $hx4$، مقدار $hx3$ را میگیرد.
\item
به این شکل که $hx3$، مقدار$ hx2$ را میگیرد.
\item
به این شکل که $hx2$، مقدار $hx1$ را میگیرد.
\item
به این شکل که $hx1$، مقدار $hx0$ را میگیرد.
\item
برای $Hx0$ اتفاق نسبتا پیچیده ای میافتد:
\begin{itemize}
\item
64 بیت کم ارزشش با 64 بیت پرارزش $h$ (ورودی) پر میشود.
\item
448 بیت پرارزشش  با بیت های $[511:64]$ از h پر میشود.
\item
64 بیت باقی مانده وسط $hx0$ ( $[64:127]$ ) با نتیجه زیر پر میشود:  (خط 341)
\begin{code}
((h[575:512] ^ h[511:448]) ^ (h[447:384] ^ h[383:320])) ^ ((h[319:256] ^ h[255:192]) ^  (h[191:128] ^ h[127: 64])) ^ 64'h1BD11BDAA9FC1A22
\end{code}

در واقع در توضیح خط بالا میتوان گفت$ xor$ تمام مجموعه های 64 بیت های ورودی $h $به جز کم ارزشترین مجموعه $( h[64:0] )$ است که در نهایت با یک عدد ثابت 64 بیتی$ xor$ شده است.

\end{itemize}
\end{itemize}
\end{itemize}

\subsection{\lr{skein\_round\_1,2,3,4}}
چهار ماژول باقی مانده با نام های
\lr{ skein\_round\_1, skein\_round\_2, skein\_round\_3 , skein\_round\_4 }
را به دلیل شباهت ساختاری با هم بررسی میکنیم.
\subsubsection{اطلاعات کلی}
\lr{
\begin{table}[H]
\centering
\begin{tabular}{@{}ll@{}}
\toprule
\multicolumn{2}{l}{skein\_round\_1,2,3,4} \\ \midrule
$clk, even, {[}511:0{]} in $   & inputs     \\
${[}511:0{]} out    $          & outputs    \\ \bottomrule
\end{tabular}
\rl{
\caption{ اطلاعات کلی ماژول‌های\lr{ skein\_round\_1,2,3,4}}}
\label{table_skein_round_n}
\end{table}}

\begin{itemize}
\item
در هر 4 ماژول دو مجموعه wire گرفته شده است:
\begin{itemize}
\item
64 بیتی : $p0, p1, p2, p3, p4, p5, p6, p7$
\item
64 بیتی :$ p0x,p1x,p2x,p3x,p4x,p5x,p6x,p7x$
\end{itemize}
\item
مجموعه ای از assignment ها داریم:
\begin{itemize}
\item
$P0$ تا $p7$ به ترتیب به 64 بیت های پرارزش تا کم ارزش in متصل اند. 
(به کم‌ارزش‌ترین $p7$)
\end{itemize}
\item
Assignment های مربوط به $p0x$ تا $p7x$ برای 4 ماژول متفاوت است در نتیجه جداگانه بررسی میکنیم:
\begin{itemize}
\item ماژول ۱\\
$Pkx$ ها با $k$ های زوج مقدار $pk + p(k+1)$ را به خود میگیرند. مثلا $p2x = p2 + p3$\\
$P1x$ با توجه به $even$ مقدار میگیرد: 
\begin{itemize}
\item
اگر $even$ یک باشد، { $p1[17:0], p1[63:18]$ } (خط شکاف بین بیت 17 و 18)
\item
اگر $even$ صفر باشد، { $p1[24:0], p1[63:25]$ } (خط شکاف بین بیت 24 و 25)
\end{itemize}
   $P3x$ با توجه به $even$ مقدار میگیرد: 
 \begin{itemize}
 \item
 اگر $even$ یک باشد، { $p3[27:0], p3[63:28]$ }  (خط شکاف بین بیت 27 و 28)
\item 
اگر $even$ صفر باشد، { $p3[33:0], p3[63:34]$ } (خط شکاف بین بیت 33 و 34)

 \end{itemize}
   $P5x$ با توجه به $even$ مقدار میگیرد:
   \begin{itemize}
   \item
   اگر $even$ یک باشد، { $p5[44:0], p5[63:45]$ }  (خط شکاف بین بیت 44 و 45)
   \item
اگر $even$ صفر باشد، { $p5[29:0], p5[63:30]$ } (خط شکاف بین بیت 29 و 30)
\end{itemize}    
   $P7x$ با توجه به $even$ مقدار میگیرد: 
   \begin{itemize}
   \item
   اگر $even$ یک باشد، { $p7[26:0], p7[63:27]$ }  (خط شکاف بین بیت 26 و 27)
   \item
اگر $even$ صفر باشد، { $p7[39:0], p7[63:40]$ } (خط شکاف بین بیت 39 و 40)

   \end{itemize}
\item ماژول ۲
\begin{code}
	//module 2 
	assign p0x = p0 + p3
	assign p1x = (even) ? { p1[30:0], p1[63:31] } : { p1[50:0], p1[63:51] }
	assign p2x = p2 + p1
	assign p3x = (even) ? { p3[21:0], p3[63:22] } : { p3[46:0], p3[63:47] }
	assign p4x = p4 + p7
	assign p5x = (even) ? { p5[49:0], p5[63:50] } : { p5[53:0], p5[63:54] }
	assign p6x = p6 + p5
	assign p7x = (even) ? { p7[36:0], p7[63:37] } : { p7[13:0], p7[63:14] }
\end{code}
با توجه به توضیحات در مورد ماژول 1، در ماژول 2 با حفظ کلیات جزئیات تغییر میکند.

\item ماژول ۳
\begin{code}
	//module 3
	assign p0x = p0 + p5
	assign p1x = (even) ? { p1[46:0], p1[63:47] } : { p1[38:0], p1[63:39] }
	assign p2x = p2 + p7
	assign p3x = (even) ? { p3[14:0], p3[63:15] } : { p3[34:0], p3[63:35] }
	assign p4x = p4 + p1
	assign p5x = (even) ? { p5[27:0], p5[63:28] } : { p5[24:0], p5[63:25] }
	assign p6x = p6 + p3
	assign p7x = (even) ? { p7[24:0], p7[63:25] } : { p7[20:0], p7[63:21] }
\end{code}
\item ماژول ۴
\begin{code}
	//module 4
	assign p0x = p0 + p7
	assign p1x = (even) ? { p1[19:0], p1[63:20] } : { p1[55:0], p1[63:56] }
	assign p2x = p2 + p5
	assign p3x = (even) ? { p3[ 7:0], p3[63: 8] } : { p3[41:0], p3[63:42] }
	assign p4x = p4 + p3
	assign p5x = (even) ? { p5[ 9:0], p5[63:10] } : { p5[ 7:0], p5[63: 8] }
	assign p6x = p6 + p1
	assign p7x = (even) ? { p7[54:0], p7[63:55] } : { p7[28:0], p7[63:29] }
\end{code}
ماژول های 3 و 4 هم در این مورد مشابهت دارند با توضیحات داده شده در مورد ماژول 1، $pkx$ های با$ k$ زوج مجموع دو $p$ هستند.  اگر $k$ فرد باشد،  $pkx$ برابر همان $pk$ خواهد بود با این تفاوت که با توجه به صفر یا یک بودن $even$ بین یکی از بیت های $pk$ شکاف میندازد و سمت چپ شکاف را در قسمت کمارزش خود و سمت راست شکاف را در قسمت پرارزش پر میکند. (در واقع همان rotation که در توصیف الگوریتم بیان شد این‌جا دیده می‌شود).
\end{itemize}
\item
تنها always-block در ماژول (حساس به لبه مثبت کلاک) :\\
در این بلاک خروجی $out$، به صورت مجموعه های 64 بیتی مقداردهی میشود. $Out$ 512 بیتی است پس 8 بار مقدار دهی لازم است. برای هر 4 ماژول مقدار های نسبت داده شده به 64 بیتی های $out$ را (به ترتیب از پرارزش ترین 64بیت تا کم ارزشترین آن) داریم :
\begin{itemize}
\item ماژول ۱
\lr{
\begin{table}[H]
\centering
\begin{tabular}{@{}lllllllll@{}}
\toprule
0                        & 1   & 2                        & 3   & 4                        & 6                        & 5   & 7   & Index \\ \midrule
$p7x \textasciicircum p6x$ & $p6x$ & $p5x \textasciicircum p4x$ & $p4x$ & $p3x \textasciicircum p2x$ & $p1x \textasciicircum p0x$ & $p2x$ & $p0x$ & Value \\ \bottomrule
\end{tabular}
\label{table_skein_round_1}
\end{table}}
\item ماژول ۲
\lr{
\begin{table}[H]
\centering
\begin{tabular}{@{}lllllllll@{}}
\toprule
0 & 1 & 2 & 3 & 4 & 6 & 5 & 7 & Index \\ \midrule
$p7x \textasciicircum p4x$ & $p6x$ & $p5x \textasciicircum p6x$ & $p4x$ & $p3x \textasciicircum p0x$ & $p2x$ & $p1x \textasciicircum p2x$ & $p0x$ & Value \\ \bottomrule
\end{tabular}
\label{table_skein_round_2}
\end{table}}
\item ماژول ۳
\lr{
\begin{table}[H]
\centering
\begin{tabular}{@{}lllllllll@{}}
\toprule
0 & 1 & 2 & 3 & 4 & 6 & 5 & 7 & Index \\ \midrule
$p7x \textasciicircum p2x$ & $p6x$ & $p5x \textasciicircum p0x$ & $p4x$ & $p3x \textasciicircum p6x$ & $p2x$ & $p1x \textasciicircum p4x$ & $p0x$ & Value \\ \bottomrule
\end{tabular}
\label{table_skein_round_3}
\end{table}}
\item ماژول ۴
\lr{
\begin{table}[H]
\centering
\begin{tabular}{@{}lllllllll@{}}
\toprule
0 & 1 & 2 & 3 & 4 & 6 & 5 & 7 & Index \\ \midrule
$p7x \textasciicircum p0x$ & $p6x$ & $p5x \textasciicircum p2x$ & $p4x$ & $p3x \textasciicircum p4x$ & $p2x$ & $p1x \textasciicircum p6x$ & $p0x$ & Value \\ \bottomrule
\end{tabular}
\label{table_skein_round_4}
\end{table}}
\end{itemize}
\end{itemize}