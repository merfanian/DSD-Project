\chapter{نتیجه‌گیری}
\noindent
\textbf{
	\textit{
	خلاصهٔ این مستند
	}
}
\pagebreak

\section{نتیجه‌گیری}
در انتها به صورت مختصر مطالبی که در این مستند بیان شد به ذکر نام و شرحی کوتاه نوشته خواهند شد.
\subsection{مقدمه}
در ابتدای مستند مقدمه‌ای کوتاه دربارهٔ کلیت الگوریتم‌های امنیتی، هدف و کاربرد ایشان نوشته شد، همچنین مختصرا به بررسی نحوهٔ کارکرد الگوریتم درهم‌سازی مورد بحث مستند پرداختیم، در انتها کاربردهای الگوریتم درهم‌سازی Skein به صورت مختصر شرح داده شد. 

\subsection{ساختار درونی سیستم}
در ادامه به شرح سخت‌افزاری سیستم و بررسی کد verilog پرداختیم، ساختار درختی و دیاگرام بلوکی در تصاویری بررسی شدند و سپس توصسف کد وریلاگ در دستور کار مستند قرار گرفت. سعی شد تا دیدگاهی عمیق‌تر از مقدمه در این قسمت موٖرد بررسی قرار بگیرد. 

\subsection{شبیه‌سازی}
در این قسمت کدهای مدل طلایی و سخت‌افزاری که توسط استاد درس در اختیار گروه قرار گرفته شده بود شبیه‌سازی و اجرا شدند، تحلیلی از هر دو کد ارائه و ایراداتی از کد وریلاگ گرفته شد. 
متاسفانه هیچ‌کدام از دو کد خروجی شبیه مرجع نداشتند و این ایراد بر مستند نوشته شده وارد است. 

\subsection{پیاده‌سازی سخت‌افزاری}
کد وریلاگ داده شده با ابزارهای سنتز بر روی FPGA مجازی سنتز شد و گزارش‌هایی از تعداد فلیپ‌فلاپ‌ها و فرکانس ارائه شد. 


\subsection{سخن پایانی}
این پروژه با هدف آشنایی گروه با نحوهٔ پیاده‌سازی یک الگوریتم صنعتی مانند الگوریتم‌های درهم‌سازی ایجاد شده بود، در طی این مسیر تجربه‌های جالبی در حوزه مستند‌سازی و نحوهٔ آنالیز سیستم‌های سخت‌افزاری برای اعضای گروه به وجود آمد. نتایج این پروژه بیشتر از جنبهٔ تجربی برای اعضای گروه قابل بررسی‌ست و متاسفانه نتیجهٔ قابل اعتنایی در حوزهٔ تحلیل سخت‌افزار به دست نیامده است؛ به عنوان مثال اعضای گروه موفق به حل برخی مشکلات کد C و وریلاگ نشدند؛ امیدواریم خوانندگان عزیز این موارد را بر ما ببخشند. با تشکر از هم‌راهی تمامی عزیزان. 