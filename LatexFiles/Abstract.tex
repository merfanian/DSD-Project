\begin{abstract}
در مستندی که پیش روی خوانندهٔ عزیز قرار دارد تلاش شده تا مختصرا الگوریتم درهم‌سازی Skein در راستای انجام پروژه درس طراحی سیستم‌های دیجیتال تشریح شود، این درس در بهار ۹۸ در دانشکده مهندسی کامپیوتر دانشگاه صنعتی شریف توسط استاد فرشاد بهاروند ارائه شده است.\\
برای سهولت کار استاد محترم درس برای تحصیل اطمینان از درستی مستندسازی و همچنین استفاده دانش‌جویان علاقه‌مند، سیر پیشرفت مستند به همراه کدهای
  	\LaTeX  \space
  در 
\textit{  \href{https://github.com/merfanian/DSD-Project}{Github} 
}  قرار گرفته است، لازم به ذکر است که این مستند به صورت متن‌باز ارائه شده و 
استفاده از آن بدون ذکر منبع برای همگان آزاد است. \\
همچنین سیر پیشرفت کل پروژه، صورتِ جلسات برگزارشده، نحوهٔ تقسیم کارها و همچنین پیش‌نویس‌هایی که منجر به ایجاد این مستند شده در 
\textit{ \href{https://drive.google.com/open?id=10nzBerJmGIVCw3dPRTYl1Ee0u9R8j8cq}{این پوشه}}
  قرار گرفته اند. در انتها از استاد محترم درس، دستیار آموزشی ایشان، خوانندگان محترم و تمامی افرادی که در تکمیل این پروژه نقش داشتند تشکر می‌کنیم.
  \flushleft
با آرزوی خوش‌وقتی برای تمامی خوانندگان این مستند\\
تیم پروژه\\
\end{abstract}