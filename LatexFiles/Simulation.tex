\chapter{شبیه‌سازی}
\noindent
\textbf{
\textit{
توصیف روند شبیه‌سازی سخت‌افزار و گام‌های اجرایی، مشاهدهٔ ورودی‌ها و خروجی‌های اصلی و میانی، مقایسه با مقادیر حاصل از اجرای کد نرم‌افزاری (مدل طلایی)، توصیف مراحل اجزای الگوریتم به همراه شکل موج‌ها، نحوهٔ عملکرد 
\lr{Testbench}
}
}
\pagebreak

\section{توضیح روند شبیه‌سازی سخت‌افزار و گام‌های اجرایی}
برای شبیه‌سازی سخت‌افزاری کد 
\lr{verilog}
الگوریتم 
\lr{Skein}
را در محیط شبیه‌سازی 
\lr{Modelsim}
اجرا کردیم. گام‌های اجرایی به صورت کلی برای شبیه‌سازی کد سخت‌افزاری موارد زیر بود. 
\begin{itemize}
\item
مطالعه کد الگوریتم و تعیین ورودی‌ها
\item
نوشتن Testbench
\item
اجرای کد در محیط Modelsim با Testbenchهای مختلف
\item
گرفتن Waveform و مقادیر خروجی (اصلی و میانی)
\end{itemize}
\section{مشاهدهٔ ورودی‌ها و خروجی‌های اصلی و میانی}
در ادامه ابتدا کد های Testbench اجرا شده بر الگوریتم و سپس Waveformهای حاصله و در انتها خروجی‌ها به صورت متنی آورده می‌شود.

\subsection{توضیح نحوهٔ عملکرد Testbench}
در ادامه ابتدا کد verilog نوشته‌شده برای هر Testbench آورده  و سپس توضیحاتی دربارهٔ آن ایراد شده است. 

 \subsubsection{\lr{Testbench 1}}
\begin{code}
//First Testbench
module top;
    reg clk = 1'b0;
    reg [511:0] midstate = 72;
    reg [95:0] data = "hello" ;
    reg [31:0] nonce = 13;
    wire [511:0] hash;
    always #1 clk = !clk;

    skein512 skein(clk, midstate , data ,nonce , hash);
    
    initial
  begin
        #300 data = "how are you?";
 		#300 data = "bye";
        #5000 $stop;
  end
endmodule
\end{code}

در این testbench ابتدا مقادیر ورودی ها ست میشود
به ترتیب به ازای clock  و midstate و data و nonce مقادیر 0 و 72 و hello و 13 ست میشوند
پس از 300 واحد زمانی مقدارdata تغییرمی کند و به how are you تبدیل میشود و پس از 300 واحد زمانی دیگر به bye تغییر میکند.
ترتیب و مقدار hash در بخش مربوط به آن آمده است.

\subsubsection{\lr{Testbench 2}}
\begin{code}
//Second Testbench
module top;
    reg clk = 1'b0;
    reg [511:0] midstate = 72;
    reg [95:0] data = "hello" ;
    reg [31:0] nonce = 23;
    wire [511:0] hash;
    always #1 clk = !clk;

    skein512 skein(clk, midstate , data ,nonce , hash);
    
    initial
  begin
        #300 data = "how are you?";
  		#300 data = "bye";
        #5000 $stop;
  end
endmodule
\end{code}

در این testbench نیز ابتدا مقادیر ورودی ها ست میشود
به ترتیب به ازای clock  و midstate و data و nonce مقادیر 0 و 72 و hello و 23ست میشوند
پس از 300 واحد زمانی مقدارdata تغییرمی کند و به how are you تبدیل میشود و پس از 300 واحد زمانی دیگر به bye تغییر میکند.
ترتیب و مقدار hash در بخش مربوط به آن آمده است
تنها تفاوت این بخش و بخش قبلی در مقادیر ورودی nonce است که از 13 به 23 تغییر داده شده است.


\subsubsection{\lr{Testbench 3}}
\begin{code}
//Third Testbench
module top;
    reg clk = 1'b0;
    reg [511:0] midstate = 72;
    reg [95:0] data = "still awake" ;
    reg [31:0] nonce = 23;
    wire [511:0] hash;
    always #1 clk = !clk;

    skein512 skein(clk, midstate , data ,nonce , hash);
    
    initial
  begin
        #300 data = "working";
        #6000 $stop;
  end
endmodule
\end{code}
در این testbench ابتدا مقادیر ورودی ها ست میشود
به ترتیب به ازای clock  و midstate و data و nonce مقادیر 0 و 72 و still awake و 23 ست میشوند
پس از 300 واحد زمانی مقدارdata تغییرمی کند وبه working تبدیل میشود.
ترتیب و مقدار hash در بخش مربوط به آن آمده است.
در این بخش با تغییر مقادیر اولیه و ثانویه data خروجی ها را با قسمت قبل مقایسه کردیم.

